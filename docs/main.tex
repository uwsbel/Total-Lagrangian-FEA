\documentclass[11pt]{article}

% Optional packages
\usepackage[utf8]{inputenc}   % for UTF-8 encoding
\usepackage{graphicx}         % for including figures
\usepackage{amsmath, amssymb, amsthm}
\usepackage{geometry}
\geometry{margin=1in}
\usepackage{enumitem}
\usepackage{hyperref}
\usepackage{xcolor}

%\definecolor{darkgreen}{rgb}{0,0.4,0}

\definecolor{arsenic}{rgb}{0.23, 0.27, 0.29}
\definecolor{charcoal}{rgb}{0.21, 0.27, 0.31}
\definecolor{hanblue}{rgb}{0.27, 0.42, 0.81}
\definecolor{blue-ncs}{rgb}{0.0, 0.53, 0.74}
\definecolor{awesome}{rgb}{1.0, 0.13,0.32}
\definecolor{darkgreen}{rgb}{0, .4,0}
\definecolor{purple}{rgb}{.55, .2,.9}

\newcommand{\firstReviewer}[1]{{\textcolor{hanblue}{{#1}}}}
\newcommand{\secondReviewer}[1]{{\textcolor{darkgreen}{{#1}}}}
\newcommand{\thirdReviewer}[1]{{\textcolor{blue-ncs}{{#1}}}}
\newcommand{\fourthReviewer}[1]{{\textcolor{purple}{{#1}}}}
%\newcommand{\updatedText}[1]{{\textcolor{red}{{#1}}}}
%\newcommand{\updatedTextB}[1]{{\textcolor{blue}{{#1}}}}
\newcommand{\updatedText}[1]{{#1}}
\newcommand{\updatedTextB}[1]{{}{#1}}


%\newcommand{\CHRONO}{{\sffamily{{Chrono}}}\xspace}
\newcommand{\SBELcomment}[1]{\textbf{\textcolor{red}{{#1}}}}
\newcommand{\todo}[1]{\textbf{\textcolor{blue}{{#1}}}}
%\newcolumntype{M}[1]{>{\centering\arraybackslash}m{#1}}
\newcommand{\SynChrono}{{\sffamily{{SynChrono}}}}


\title{Iterative Stochastic Gradient-Based Total Lagrangian Formulation for Flexible Multi-Body Systems}
\author{SBEL Team \\ University of Wisconsin-Madison}
\date{\today} % or leave empty with \date{}

\begin{document}

\maketitle

\begin{abstract}
This document outlines the design of a Total Lagrangian finite element solver in Project Chrono, targeting GPU acceleration and first-order optimization in a stochastic setting. The solver frames each time step as an optimization problem whose first-order optimality conditions recover the equations of motion. Optimization is performed using stochastic gradient methods with momentum, combined with Augmented Lagrangian Methods (ALM) for handling bilateral constraints and contact/friction.
\end{abstract}

\section{Highlights of the Approach}
\begin{itemize}
    \item The parallelism is on a per nodal unknown basis. 
    \begin{itemize}
        \item Tasks that appear on a per-element basis parallel will be cast to be on a per-nodal unknown basis.
    \end{itemize}
    \item The solver is a minimization-based solver. It does not use Hessian information. It is gradient-based, first order.
    \item We favor Nesterov or AdamW. To the extent possible, we want to avoid global synchronizations.
    \item We are going to use coopreative threads. As such, the entire grid should be device resident. This places a constraint on the grid size.
    \item The approach is stochastic because system-level synchronization will not be necessarily enforced at each iteration, and a thread might work with slightly stale neighboring states (which nonetheless get continuously updated). At each time step, we might “system-level” synchronize after 100 Nesterov iterations, then every 50 iterations, then 10, and close to convergence at each iteration. Each iteration will be exceptionally fast and embarrassingly parallel: a GPU thread only evaluates one equation of motion residual and updates its state in shared and/or global memory. The 100, 50, 10 values are exemplary – it remains to investigate how to tune online the frequency of the system-level synchronizations to accelerate convergence
\end{itemize}


\section{Core Concepts}
\subsection{Discretization and State}
\begin{itemize}[leftmargin=*]
\item \textbf{Kinematics}: Total Lagrangian (reference-space quantities fixed).
\item \textbf{Precompute (CPU)}: shape function gradients $\nabla_0 N$, quadrature weights, reference volumes.
\item \textbf{State}: generalized positions $q$, velocities $v$, multipliers $\lambda$, ALM penalties $\mu$.
\end{itemize}


\subsection{Variational Time Stepping}
\SBELcomment{Add the variational time stepping objective here.}

\subsection{First-Order Optimizers}
\begin{itemize}[leftmargin=*]
\item \textbf{Nesterov Accelerated Gradient (NAG)}
\item \textbf{Heavy-Ball (Polyak)}
\item \textbf{AdamW} with decoupled weight decay, acting as damping.
\end{itemize}
Each optimizer is applied in a fully parallel fashion, thread-per-nodal unknown.


\subsection{Asynchronous / Stochastic Execution}
\begin{itemize}[leftmargin=*]
\item Each thread owns one nodal unknown (might have two own two or three nodal unknowns, for very large problems; this is because we have to fit the grid size into the GPU).
\item Threads read neighboring states; values may be stale.
\item Periodic synchronization ensures system-level coherence.
\end{itemize}


\subsection{Constraints via ALM}
\begin{itemize}[leftmargin=*]
\item \textbf{Equalities}: $c(q) = 0$.
\item \textbf{Contact}: $g_n(q) \geq 0$.
\item \textbf{Friction}: proximal projection onto the Coulomb cone. After each update of the contact/friction multipliers, snap them back into the admissible Coulomb cone via a simple closed-form projection.
\item \textbf{ALM Update}: $\lambda \leftarrow \lambda + \mu h(q)$, with $h(q)$ the violation.
\end{itemize}


\section{GPU Implementation Notes}
\begin{itemize}[leftmargin=*]
\item Structure-of-Arrays (SoA) layout for $q, v, m$.
\item Kernels: element residuals, scatter to nodes, primal update, contact detection, constraint updates.
\item Use FP32 for element math; accumulate sampled energy in FP64.
\item Atomics acceptable in FP32; shared memory staging for hot spots.
\end{itemize}


\section{Synchronization Strategy}
\begin{itemize}[leftmargin=*]
\item Start with global sync every 100 iterations.
\item Reduce to 50, then 10, finally every iteration near convergence.
\item Sync triggers: plateau in sampled residuals, sentinel constraint stagnation, excessive displacement.
\end{itemize}


\section{Mini-Batching}
\begin{itemize}[leftmargin=*]
\item Batch over elements for internal forces.
\item Batch over constraints for ALM.
\item Interleave batches to reduce drift (e.g., 2:1 ratio).
\end{itemize}


\section{Stability Enhancements}
\begin{itemize}[leftmargin=*]
\item Mass-scaled step sizes $\alpha_i \propto 1/(m_i + \eta k_i \Delta t^2)$.
\item Gradient clipping at per-block percentiles.
\item Light Rayleigh damping to suppress chatter.
\end{itemize}


\section{Why this Approach?}
When advancing many small time steps (e.g., $\Delta t = 10^{-4}$), it is inefficient to launch one GPU kernel per step. Instead, a persistent cooperative kernel can execute hundreds of time steps on-device before returning control to the host, reducing overhead and keeping all state resident in GPU memory.

\section{Approach}
\subsection{Persistent Cooperative Kernel}
\begin{itemize}[leftmargin=*]
  \item Launch once with \texttt{cudaLaunchCooperativeKernel} so that grid-wide synchronization is available.
  \item Inside the kernel, advance $N$ time steps (e.g., 100) using a device-side loop.
  \item Each iteration computes element residuals, constraint updates, and optimizer steps.
  \item Periodically use \texttt{grid.sync()} for system-level coherence.
\end{itemize}

\subsection{Synchronization Strategy}
\begin{itemize}[leftmargin=*]
  \item Perform global syncs infrequently: every 100, 50, 10, then 1 iteration near convergence.
  \item At sync points:
    \begin{itemize}
      \item Refit/rebuild BVHs for contact.
      \item Update active sets for contact/friction.
      \item Publish snapshots for host output.
    \end{itemize}
\end{itemize}

\subsection{Occupancy and Kernel Design}
\begin{itemize}[leftmargin=*]
  \item Cooperative kernels require all blocks resident simultaneously; configure grid size accordingly.
  \item Use grid-stride loops so persistent blocks can cover arbitrary problem sizes.
  \item Keep optimizer state (e.g., AdamW $m,v$) in SoA arrays; minimize shared memory usage.
  \item Stage element contributions in shared memory, then scatter with atomics.
\end{itemize}

\subsection{Watchdog Considerations}
\begin{itemize}[leftmargin=*]
  \item On Windows WDDM, long kernels risk TDR termination ($\sim$2s).
  \item Mitigations:
    \begin{itemize}
      \item Use TCC-mode GPUs or WSL compute mode.
      \item Chunk into smaller launches (10--50 ms wall time each) and chain with CUDA Graphs.
    \end{itemize}
\end{itemize}

\subsection{Output Handling}
\begin{itemize}[leftmargin=*]
  \item Device tracks physical time; output triggered when $t \geq t_{\text{next}}$.
  \item Snapshots staged into global memory; one warp or block writes, others wait at grid-sync.
  \item Optional: asynchronous copies via pinned buffers and host polling for streaming output.
\end{itemize}

\section{Benefits}
\begin{itemize}[leftmargin=*]
  \item Eliminates launch overhead for each time step.
  \item Keeps data on-GPU, improving locality.
  \item Matches stochastic first-order optimization: most iterations are asynchronous, with coherence enforced only at sync points.
  \item Provides flexibility for output scheduling (e.g., $\Delta t = 10^{-4}$, advance from 2.36 s to 2.37 s with 100 steps before host handoff).
\end{itemize}

\section{Timeline}
\begin{enumerate}[leftmargin=*]
\item \textbf{Weeks 1--2}: Element library, TL kinematics, gradient kernel, BB + backtracking.
\item \textbf{Week 3}: Variational time stepping objective, inertial prediction, patch tests.
\item \textbf{Weeks 4--5}: Contact detection (BVH broadphase, narrowphase), unilateral ALM.
\item \textbf{Week 6}: Friction via prox projection, sliding tests.
\item \textbf{Week 7}: Bilateral constraints (joints) as ALM equalities.
\item \textbf{Week 8}: Many-body tests, constraint coupling, performance profiling.
\item \textbf{Week 9+}: Adaptive synchronization, parameter auto-tuning, mixed precision optimization.
\end{enumerate}


\section{Conclusion}
Using a persistent cooperative kernel to integrate many time steps per host launch aligns well with a stochastic, gradient-based TL-FEA solver. The design reduces overhead, maximizes GPU utilization, and allows coarse-grained synchronization for contact and constraint updates.

\bibliographystyle{ieeetr} % or another style
\bibliography{BibFiles/refsMBS} % assumes references.bib

\end{document}


\end{document}
